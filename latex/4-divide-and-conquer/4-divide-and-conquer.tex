\documentclass[a4paper,12pt,fleqn]{article}

% Page layout
\usepackage[width=15.5cm]{geometry}

% Use paragraph spacing, not indents
\usepackage{parskip}

% Language shenanigans
\usepackage[english]{babel}

% Sane math typesetting
% NOTE: keep this above the fontspec/unicode-math packages
% loads amsmath as a dependency
\usepackage{mathtools,amssymb,amsthm}

% Algorithm typesetting
\usepackage[%
    ruled,
    linesnumbered
]{algorithm2e}

% NOTE: Keep these uses below any font/math related packages
% Need to install the fonts, or comment them out if you don't care
\usepackage{fontspec}
\setmainfont{STIX Two Text}
\setmonofont[Scale=0.88]{Iosevka}
% NOTE: Keep unicode-math below fontspec
\usepackage{unicode-math}
\setmathfont{STIX Two Math}

% Easy to use colors
\usepackage[dvipsnames]{xcolor}

% Better referencing with some sane defaults
% Keep this below all packages
\usepackage[%
    % make the text + page nums link to their resp. content in TOC,LOT,LOF
    linktoc=all,
    % color the text of links and anchors
    colorlinks=true,
    % set colors for internal, citation, URL links
    linkcolor=violet!85!black,
    citecolor=YellowOrange!85!black,
    urlcolor=Aquamarine!85!black
]{hyperref}


% ============================
% Custom commands/environments
% ============================

% Solution texts
% Invocation: \exsoln{solution_number}{question_text}{answer_text}
\newcommand{\exsoln}[3]{%
» \textbf{\emph{Exercise}} \emph{#1}% Solution number

\emph{#2}% Question text

\textbf{\emph{Solution:}}\\
#3% Answer text

\bigskip % ideally wouldn't insert this unless another exercise solution follows
}

% Array slice
% Invocation: \arrslice{varname}{startindex}{endindex}
\newcommand{\arrslice}[3]{%
\ensuremath{\mathtt{#1[#2\ldots #3]}}
}

% Array index
% Invocation: \arrat{varname}{index}
\newcommand{\arrat}[2]{%
\ensuremath{\mathtt{#1[#2]}}
}


\begin{document}
% === TITLE === %
\title{CLRS Chapter 4 \\ Divide and Conquer}
\author{Md Istiaque Al Jobayer}
\date{10th September 2021}
\maketitle

% === CONTENT === %
\section{Maximum Subarray CLRS-4.1}
\exsoln{4.1-1}
{What does \texttt{FIND-MAXIMUM-SUBARRAY} return when all elements of \textit{A} are negative?}
{It returns the largest number of \textit{A}.}

\exsoln{4.1-2}
{Write pseudocode for the brute-force method of solving the maximum subarray problem. Your procedure should run in $\Theta(n^2)$ time.}{%
    Refer to Algorithm \ref{algo:max-subarray-naive}. The idea is quite straightforward: try all possible subarrays \arrslice{A}{i}{j} and iteratively update the maximum found till all subarrays are processed and only a global maximum remains. The running time is $\Theta(n^2)$ since we try all $\binom{n}{2}$ pairs.
    \begin{algorithm}
        \caption{\texttt{MAX-SUBARRAY-NAIVE}}
        \label{algo:max-subarray-naive}

        % data variables
        \SetKwData{maxSum}{maxSum}
        \SetKwData{sumSoFar}{sumSoFar}
        \SetKwArray{A}{A}

        % functions
        \SetKwFunction{max}{max}

        % global settings
        \DontPrintSemicolon

        % input/output
        \KwIn{an array \A{$1 \ldots N$} of totally ordered objects with $<$ defined}
        \KwOut{\maxSum, the largest subarray sum in the array}
        \BlankLine

        % the algorithm
        $\maxSum = -\infty$\;

        \For{$i = 1$ \KwTo $N$}{%
            \sumSoFar$= 0$\;
            \For{$j = i$ \KwTo $N$}{%
                $\sumSoFar=\sumSoFar+\A{\,j}$\;
                $\maxSum = \max{\sumSoFar, \maxSum}$\;
            }
        }

        \Return{\maxSum}
    \end{algorithm}
}

\exsoln{4.1-3}
{Implement both the brute-force and recursive algorithms for the maximum subarray problem on your own computer. What problem size $n_0$ gives the crossover point at which the recursive algorithm beats the brute-force algorithm? Then, change the base case of the recursive algorithm to use the brute-force algorithm whenever the problem size is less than $n_0$. Does that change the crossover point?}
{The crossover point found is $n_0 = 40$. Changing it forces the crossover point to $n_0 = 50$. Refer to the file \texttt{maxsubarray.py} for details.}

\exsoln{4.1-4}
{Suppose we change the definition of the maximum subarray problem to allow the result to be an empty subarray, where the sum of the values of such a subarray is $0$. How would you change any of the algorithms that do not allow empty subarrays to permit them?}
{Interpret $\textrm{low} > \textrm{high}$ as empty subarray, returning 0 for that case.}

\exsoln{4.1-5}
{Use the following ideas to develop a non-recursive, linear time algorithm for the maximum subarray problem. Start at the left end of the array, progess toward the right, keeping track of the maximum subarray seen so far. Knowing a maximum subarray of \arrslice{A}{1}{j}, extend the answer to find the maximum subarray ending at index $j+1$ by using the following observation: a maximum subarray of \arrslice{A}{1}{j+1} is either a maximum subarray of \arrslice{A}{1}{j} or a maximum subarray \arrslice{A}{i}{j+1}, for some $1 \leq i \leq j+1$. Determine a maximum subarray of the form \arrslice{A}{i}{j+1} in constant time based on knowing a maximum subarray ending at index $j$.}{%
    Refer to Algorithm \ref{algo:max-subarray-linear}. The key idea here is that given the following descriptions:
    \[ M(k) = \text{a maximum subarray sum of }\arrslice{A}{1}{k} \]
    \[ E(k) = \text{maximum subarray sum ending at and including index }\mathtt{k} \]
    we can characterize a solution to $M(k)$ by observing the following relationships:
    \[ M(k) = \max(M(k-1), E(k)) \]
    \[ E(k) = \max(\arrat{A}{k}, E(k-1)+\arrat{A}{k}) \]

    To elaborate a bit further, the maximum subarray of \arrslice{A}{1}{k} either includes \arrat{A}{k}, or does not. If it does not, that is $M(k-1)$. If it does, then it is the maximum ending at that index: $E(k)$. Furthermore, $E(k)$ by itself is also either growing the previous subarray obtained from $E(k-1)$, or we begin a new subarray containing just \arrat{A}{k}.
    \begin{algorithm}
        \caption{\texttt{MAX-SUBARRAY-LINEAR}}
        \label{algo:max-subarray-linear}

        % data
        \SetKwData{globalMax}{globalMax}
        \SetKwData{maxEndingHere}{maxEndingHere}
        \SetKwArray{A}{A}

        % functions
        \SetKwFunction{max}{max}

        % global settings
        \DontPrintSemicolon

        % input/output
        \KwIn{array \A{$1\ldots N$} of totally ordered objects with $<$ defined}
        \KwOut{\globalMax, a maximum subarray sum}

        % the algorithm
        \BlankLine
        $\globalMax = -\infty$\;
        $\maxEndingHere = 0$\;
        \For{$k = 1$ \KwTo $N$}{%
            $\maxEndingHere = \max{\A{k}, \maxEndingHere + \A{k}}$\;
            $\globalMax = \max{\globalMax, \maxEndingHere}$\;
        }
        \Return{\globalMax}
    \end{algorithm}
}

\end{document}